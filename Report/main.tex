% ****** Start of file apssamp.tex ******
%
%   This file is part of the APS files in the REVTeX 4.1 distribution.
%   Version 4.1r of REVTeX, August 2010
%
%   Copyright (c) 2009, 2010 The American Physical Society.
%
%   See the REVTeX 4 README file for restrictions and more information.
%
% TeX'ing this file requires that you have AMS-LaTeX 2.0 installed
% as well as the rest of the prerequisites for REVTeX 4.1
%
% See the REVTeX 4 README file
% It also requires running BibTeX. The commands are as follows:
%
%  1)  latex apssamp.tex
%  2)  bibtex apssamp
%  3)  latex apssamp.tex
%  4)  latex apssamp.tex
%
\documentclass[%
 reprint,
%superscriptaddress,
%groupedaddress,
%unsortedaddress,
%runinaddress,
%frontmatterverbose, 
%preprint,
%showpacs,preprintnumbers,
%nofootinbib,
%nobibnotes,
%bibnotes,
 amsmath,amssymb,
 aps,
%pra,
%prb,
%rmp,
%prstab,
%prstper,
%floatfix,
]{revtex4-1}

\usepackage{subcaption}
\usepackage{graphicx}% Include figure files
\usepackage{dcolumn}% Align table columns on decimal point
\usepackage{bm}% bold math
\usepackage{listings}
\usepackage{float}
\usepackage[bottom]{footmisc}
%\usepackage{hyperref}% add hypertext capabilities
%\usepackage[mathlines]{lineno}% Enable numbering of text and display math
%\linenumbers\relax % Commence numbering lines

%\usepackage[showframe,%Uncomment any one of the following lines to test 
%%scale=0.7, marginratio={1:1, 2:3}, ignoreall,% default settings
%%text={7in,10in},centering,
%%margin=1.5in,
%%total={6.5in,8.75in}, top=1.2in, left=0.9in, includefoot,
%%height=10in,a5paper,hmargin={3cm,0.8in},
%]{geometry}


\begin{document}
\lefthyphenmin=62
\righthyphenmin=62
%\preprint{APS/123-QED}

\title{}

\author{Loes Erven (s4538757)}
\author{Leander van den Heuvel (s...)}


\date{\today}% It is always \today, today,
             %  but any date may be explicitly specified

\begin{abstract}   
\end{abstract}

6 to 8 pages:
 title.
 An abstract (about 100 words) 
 
In the body of the report:
	intro:
		the application domain and the research problem;
		related previous work that attempted to address the same or a similar problem;
    experiment, methods:
		your data set and data collection, preprocessing methods;
		your approach/method(s) and, often more importantly, the rationale behind your decisions;
    results
    	the main results of your project;
    discussion/conclusion:
		an analysis of results and evaluation of your project outcome: to what extent it matches your expectation and why if there is a gap;
	general discussion
		possible future directions to improve the results;
	appendix:
		in the appendix, add a listing of the (main) files that you used to generate the results.
        
Also clearly identify any innovative ideas that you explored in your project, by comparing related work to the best of your knowledge.


\maketitle

%\tableofcontents

\section{}



\end{document}








